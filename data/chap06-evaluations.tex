\chapter{实验与测试}
\label{cha:evaluations}

对上面三章所提的内容,本章将开展四个实验进行综合测试和分析。
首先是GPU渲染器与CPU渲染器之间的性能比较测试,该实验将在同一场景下分别运行本人实现的GPU渲染器和PBRT渲染器(基于CPU,可支持多线程),
通过比较渲染时间及效果探讨该GPU渲染器的效率优劣;
其次是渲染算法实验,该实验将利用GPU渲染器对第四章中的部分渲染算法开展实验,以探究本GPU渲染器在研究渲染算法中的具体使用价值;
最后,将围绕第五章中介绍的VRay模拟材质的进行效果测试,并尝试总结其达到的模拟程度以及与VRay原材质之间的主要差异。

以下实验均是在RTX 2080显卡+(TODO:CPU型号)的硬件环境下完成的。

\section{GPU渲染与CPU渲染的性能比较实验}

为了体现测试的公平性,首先需要做出适当的规定,以保证双方渲染算法的一致性。在该实验中,
双方需要运行不包含任何优化的路径追踪算法,以输出分辨率为720p的图像;渲染的场景只能包含Disney BRDF材质,
理由是该材质有着精确的定义,且在双方的渲染器中都有实现;渲染中所使用的基本参数如相机参数、光线最大深度都必须完全一致。
在以上的这些限制下,两款渲染器需要各自运行1000次采样流程,并记录总共花费的时间以进行对比。

经过筛选,本人最终选择了三个复杂度各异的场景用来测试:Bedroom、TODO、TODO。对于GPU渲染器,
将记录其在开启显卡的RTX功能下的运行速度和没有开启RTX时的运行速度;而对于CPU渲染器,
则会依次记录其在单线程、双线程、四线程和八线程下的运行效率。以下是运行的结果:

可以看到(TODO:这里有结果了再补充)。以上的数据足以说明,GPU对于光线追踪渲染的效率提升能够起到显著的作用。
这也从侧面证明了本渲染器的使用价值。

\section{渲染算法实验}
该实验共由两个子实验构成:
第一个子实验是对\ref{adaptive}小节中提到的自适应采样算法进行测试;
第二个则是对路径追踪算法和SPPM算法的综合比较。

\subsection{路径追踪算法的自适应采样测试}


\subsection{路径追踪算法与SPPM算法比较}

\section{VRay模拟材质效果测试}