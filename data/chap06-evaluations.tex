\chapter{实验与测试}
\label{cha:evaluations}

对上面三章所提的内容,本章将开展四个实验进行综合测试和分析。
首先是GPU渲染器与CPU渲染器之间的性能比较测试,该实验将在同一场景下分别运行本人实现的GPU渲染器和PBRT渲染器(基于CPU,可支持多线程),
通过比较渲染时间及效果探讨该GPU渲染器的效率优劣;
其次是渲染算法实验,该实验将利用GPU渲染器对第四章中的部分渲染算法开展实验,以探究本GPU渲染器在研究渲染算法中的具体使用价值;
最后,将围绕第五章中介绍的VRay模拟材质的进行效果测试,并尝试总结其达到的模拟程度以及与VRay原材质之间的主要差异。

以下实验均是在RTX 2080显卡+(TODO:CPU型号)的硬件环境下完成的。

\section{GPU渲染与CPU渲染的性能比较实验}

为了体现测试的公平性,首先需要做出适当的规定,以保证双方渲染算法的一致性。在该实验中,
双方需要运行不包含任何优化的路径追踪算法,以输出分辨率为720p的图像;渲染的场景只能包含Disney BRDF材质,
理由是该材质有着精确的定义,且在双方的渲染器中都有实现;渲染中所使用的基本参数如相机参数、光线最大深度都必须完全一致。
在以上的这些限制下,两款渲染器需要各自运行1000次采样流程,并记录总共花费的时间以进行对比。

经过筛选,本人最终选择了三个复杂度各异的场景用来测试:Bedroom、TODO、TODO。对于GPU渲染器,
将记录其在开启显卡的RTX功能下的运行速度和没有开启RTX时的运行速度;而对于CPU渲染器,
则会依次记录其在单线程、双线程、四线程和八线程下的运行效率。以下是运行的结果:

可以看到(TODO:这里有结果了再补充)。以上的数据足以说明,GPU对于光线追踪渲染的效率提升能够起到显著的作用。
这也从侧面证明了本渲染器的使用价值。

\section{渲染算法实验}
该实验共由两个子实验组成:
首先会对\ref{adaptive}小节中提到的自适应采样算法进行测试;
其次则是对路径追踪算法和SPPM算法的综合比较。

\subsection{路径追踪算法的自适应采样测试}

为了验证自适应采样算法在本渲染器中的实际运行效果,本人设计了如下的实验方案:
首先,使用该渲染器对制定的场景运行路径渲染算法,将渲染足够长时间后(实验中设定为2000次采样)得到的结果视为参考答案(Reference);
接着,对同样的场景运行携带不同参数的自适应路径追踪算法,
记录其结果与参考答案之间的误差随渲染时间变化的趋势,然后与一般的路径渲染算法进行对比。
此外,为了消除随机数所带来的影响,每种算法都会采用不同的随机种子运行多次(实验中为5次),然后对各次的误差取平均值作为结果。

在\ref{adaptive}小节介绍的自适应采样算法中,共有两个参数可以进行修改:$\varepsilon_s, \varepsilon_t$。
在本实验中只对$\varepsilon_t$进行调整,而规定$\varepsilon_s = 256\cdot \varepsilon_t$。

在数据上,本实验采用了Sponza、Bedroom以及TODO三个不同类型的场景分别进行测试。下面的图表显示了最终实验结果:

TODO:自适应采样实验结果(这里需要采用误差还是偏差)

从图表中可以看出,搭载了自适应采样的渲染器从渲染中期开始便逐渐取得了更快的收敛速度,
直到算法宣布渲染终止前都能取得相较普通路径追踪更好的结果。
这和算法提出者对于该算法的描述是一致的。

\subsection{路径追踪算法与SPPM算法比较}

路径追踪算法与SPPM算法的目的都是求解渲染方程,因此这两者最后的理论收敛值应当是完全一致的。
我们采用和上一章中一样的实验方法,即先通过路径追踪算法得到参考结果,再分别运行两个算法,记录误差随时间变化的情况。
实验结果如下图所示:

TODO:实验结果图。

除去误差的不同外,两者在生成的图像上看也通常具有一些差异。
下表(TODO:结果图)中包含了两种算法在运行过程中的部分截图,
可以看出,SPPM算法图像中的噪声相比路径追踪算法要柔和许多,但在一些特殊区域如角落却显得十分阴暗。
这些差异都是与SPPM算法的实际特性相符合的。

\section{VRay模拟材质效果测试}

关于VRay模拟材质,由于缺少精确的VRay材质数据,且VRay渲染其对渲染结果已经做出了一些后处理
,所以该实验只从视觉效果上对双方渲染的结果进行对比,没有进行量化的比较。下表是双方的一些渲染图:

TODO:VRay效果图

需要说明的是,有些结果图上的差异其实是由于双方渲染器中光源、摄像机等属性的实现方式不同而导致的,
因此与材质无关。
总体来看,VRay模拟材质的渲染效果已经和VRay原材质比较接近,
但在一些细节上(如半透明等)还存在很大的偏差。

