\chapter{基于GPU的光线追踪渲染}
\label{cha:algorithms}

上一章介绍了渲染器的整体架构以及Host程序中的各种关键类,
在此基础上,接下来则分析如何用OpTiX中的GPU程序实现具体的渲染算法。
本人先会对需要实现的算法原理、公式进行简单介绍,然后逐步解释算法中的各个步骤是怎样完成的。

\section{路径追踪算法}
路径追踪算法的原理即利用蒙特卡洛积分近似求解渲染方程(式2-1)。
简单来说,当光线在场景中与物体相交时,
我们可以按照一定的概率密度函数$p(\omega)$对次级光线的角度$\omega_i$进行采样,
然后通过下式来近似求得该光线所得的出射光照亮度:
\begin{equation}
L_o(\textbf{x}, \omega_o) 
=\frac{1}{N}\sum_{k=1}^{N}\frac{f(\omega_i)}{p(\omega_i)} 
=\frac{1}{N}\sum_{k=1}^{N}\frac{L_i(\textbf{x}, \omega_i)f(\textbf{x}, \omega_o, \omega_i)|cos\theta_i|}{p(\omega_i)}
\end{equation}
其中$N$为采样个数,$L_i$为入射光照亮度,而$f$则为BRDF函数。为求方便,通常会用$\beta_i=\frac{f(\textbf{x}, \omega_o, \omega_i)|cos\theta_i|}{p(\omega_i)}$直接代表$L_o$和$L_i$之间的系数。
对于多次散射组成的路径而言,我们只需要计算出路径中每次采样对应的$\beta$之乘积,
然后再与路径终点处的光照强度值相乘,便可以获得路径起点处的光照强度估计值了。

\subsection{算法流程}
\label{cha:pathtrace}

为了方便描述,我们先将场景中的光源简化为单一点光源,其位置为$\textbf{x}_{lt}$,辐射亮度为$L_e$。

在上面的公式中,一共有三个部分需要通过材质而求得:它们分别是采样值$\omega_i$、
采样的概率密度值$p(\omega_i)$以及BRDF值$f(\textbf{x}, \omega_o, \omega_i)$。
为此,我们对MaterialProperty添加三个相应的计算函数:Sample函数,Pdf函数和Eval函数。
他们的输入包含经过填充的材质属性PaddingMaterialProperty,相交点的信息(位置、向量等)以及当前光线所携带的其它信息。

我们首先定义OpTiX中使用的Closest Hit Program。对于非光源物体而言,该函数的流程则如算法\ref{CHPPath}所示:
\begin{algorithm}
    \SetKwInOut{KIN}{输入}
    \SetKwInOut{KDATA}{输出}
    \caption{路径追踪Closest Hit Program}
    \label{CHPPath}
    \KIN{相交点位置及法向量$\textbf{x},\textbf{n}$,出射光线角$\omega_o$,系数乘积$\beta$,光照亮度$L_o$}
    \KDATA{入射光线起点$\textbf{x}_i$,入射光线方向$\omega_i$,系数乘积$\beta$,光照亮度$L_o$}

    获取物体的材质属性$mat$(PaddingMaterialProperty);\\
    根据材质的种类,在保存函数ID的Buffer中取出对应的三个计算函数;\\
    \If {$\textbf{x}$与$\textbf{x}_{lt}$之间没有其它物体遮挡} 
    {
        $\omega_{lt} \leftarrow \frac{\textbf{x}_{lt}-\textbf{x}}{|\textbf{x}_{lt}-\textbf{x}|}$;\\
        $L_o \leftarrow L_o+\beta\cdot L_e\cdot {\rm Eval}(mat, \textbf{n}, \omega_o, \omega_{lt}) $;
    }
    $\omega_i \leftarrow {\rm Sample}(mat, \textbf{n}, \omega_o)$;\\
    $\textbf{x}_i \leftarrow \textbf{x}$;\\
    $\beta \leftarrow \beta\cdot {\rm Eval}(mat, \textbf{n}, \omega_o, \omega_i)/ {\rm Pdf}(mat, \textbf{n}, \omega_o, \omega_i)$;\\
\end{algorithm}

关于算法中的第3步算法没有展开详细说明,这里补充一下:
具体的实现方法是生成阴影线(Shadow Ray)并调用rtTrace对其跟踪,然后通过Any Hit Program来判断该阴影线是否被遮挡。
另外,由于该阶段所采用的光源均为点光源,因此这里先暂不考虑光源物体的Hit Program。

接下来,为了描述Ray Generation Program,我们需要对rtTrace的格式进行定义:
我们规定rtTrace以光线信息(起点、方向)和用户自定义输入作为输入,而以次级光线信息和用户自定义输出作为输出。
其中,用户自定义输入(出)的格式应当和Closest Hit Program中的输入输出(除由OpTiX生成的输入外)保持一致。

另外,由于这里要通过相机生成初始光线,因此需要在CameraProperty中添加一个计算函数CameraSample。
该函数以相机参数及运行时的索引作为输入,而以光线的起点$x$、方向$\omega_o$作为输出。

Ray Generation Program的实现见算法\ref{RGPPath},至此路径追踪算法最重要的核心部分均以完成。
\begin{algorithm}
    \SetKwInOut{KIN}{输入}
    \SetKwInOut{KDATA}{数据}
    \SetKwInOut{KOUT}{输出}
    \caption{路径追踪Ray Generation Program}
    \label{RGPPath}
    \KIN{索引$index_{x,y}$}
    \KDATA{累计平均光照亮度$L_{acc}$,总采样数$S$}
    \KOUT{光照亮度$L_{out}$}

    获取相机属性$cam$(PaddingCameraProperty);\\
    根据相机种类,取出对应的CameraSample函数;\\
    $(\textbf{x}, \omega_o) \leftarrow {\rm CameraSample}(cam, index_{x,y})$;\\
    $\beta \leftarrow 1$;\\
    $L_o \leftarrow 0$;\\
    $d \leftarrow 0$;\\
    \For{$d < d_{max}$}
    {
        $(\textbf{x}, \omega_o, \beta, L_o) \leftarrow {\rm rtTrace}(\textbf{x}, \omega_o, \beta, L_o)$;\\
        $d \leftarrow d+1$; \\
    }
    $L_{acc} \leftarrow (L_o+L_{acc}\cdot S)/(S+1)$;\\
    $L_{out} \leftarrow L_{acc}$;\\
    $S \leftarrow S+1$;
\end{algorithm}

\subsection{光照计算}

之前为了方便我们对场景中的光源提出了限定,这一节的内容则是讨论如何将这一限定去除。

首先我们需要回顾一下在第二章提到过的复合重要性采样算法。
该算法的基本思路是,在对一个函数$f$使用蒙特卡洛积分法求积分时,
我们可以采用多种不同的采样策略(即不同的概率密度函数)进行采样,然后按照下式对所有策略的采样结果进行合并:
\begin{equation}
    \sum_{s=1}^{M} \sum_{i=1}^{N_s}\frac{f(X_{si})w_s(X_{si})}{p_s(X_{si})}
\end{equation}
其中$w_s$是一个权重函数,其值等于:
\begin{equation}
    w_s(x) = \frac{(N_sp_s(x))^\beta}{\sum_i(N_ip_i(x))^\beta}
\end{equation}
$\beta$是一个固定参数,一般设为2。利用这个方法,路径追踪算法便可以方便地渲染含有面光源的场景了:
对于与物体相交的光线,采用两种采样策略来计算直接光照:第一种策略是按照光源的能量分布在光源表面采样,然后计算采样点向相交点传输的辐射亮度;
第二种策略则是按照物体表面的BRDF进行次级光线的采样,然后计算光源通过次级光线向物体传输的辐射亮度。两种策略我们都只进行一次采样。

对于第一种策略,我们通过对Closest Hit Program中的第3-6步进行修改来完成。
由于要对光源进行采样,所以LightProperty需要添加LightSample函数和LightPdf函数。
这里要注意的是,LightSample函数的采样结果是点而非方向。另外,还需要加入LightEval函数,用来求光源上某点朝某个方向输出的辐射亮度。
具体的步骤详见算法\ref{Light1}。
第二种策略则通过修改光源的Closest Hit Program来实现,详见算法\ref{Light2}:

\begin{algorithm}
    \SetKwInOut{KIN}{输入}
    \SetKwInOut{KDATA}{输出}
    \caption{光照计算(策略一)}
    \label{Light1}
    \KIN{相交点位置及法向量$\textbf{x},\textbf{n}$,出射光线角$\omega_o$,系数乘积$\beta$,光照亮度$L_o$}
    \KDATA{光照亮度$L_o$}

    获取光源属性$light$(PaddingLightProperty);\\
    根据光源的种类,取出对应的三个计算函数;\\
    $(\textbf{x}_{lt}, \textbf{n}_{lt}) \leftarrow {\rm LightSample}(light)$;\\
    $\omega_{lt} \leftarrow \frac{\textbf{x}_{lt}-\textbf{x}}{|\textbf{x}_{lt}-\textbf{x}|}$;\\
    $p_{lt} \leftarrow {\rm LightPdf}(light, \textbf{x}_{lt})$;\\
    $p_m \leftarrow {\rm Pdf}(mat, \textbf{n}, \omega_o, \omega_{lt})$;\\
    $w_l \leftarrow \frac{p_{lt}^2}{(p_{lt}^2+p_m^2)}$;\\
    \If {$\textbf{x}$与$\textbf{x}_{lt}$之间没有其它物体遮挡} 
    {$L_o \leftarrow L_o+\beta\cdot(w_l/p_{lt})\cdot {\rm LightEval}(light, \textbf{n}_{lt}, \omega_{lt})\cdot {\rm Eval}(mat, \textbf{n}, \omega_o, \omega_{lt}) $;} 
\end{algorithm}
\begin{algorithm}
    \SetKwInOut{KIN}{输入}
    \SetKwInOut{KDATA}{输出}
    \caption{光照计算(策略二)}
    \label{Light2}
    \KIN{相交点位置及法向量$\textbf{x},\textbf{n}$,出射光线角$\omega_o$,系数乘积$\beta$,光照亮度$L_o$,上轮对材质采样的概率密度$p_m$}
    \KDATA{光照亮度$L_o$}

    获取光源属性$light$(PaddingLightProperty);\\
    根据光源的种类,取出对应的三个计算函数;\\
    $p_{lt} \leftarrow {\rm LightPdf}(light, \textbf{x})$;\\
    \eIf {无可用的$p_m$} 
    {
        $L_o \leftarrow L_o+\beta\cdot{\rm LightEval}(light, \textbf{n}_{lt}, \omega_{lt}) $;
    } 
    {
        $w_l \leftarrow \frac{p_m^2}{(p_{lt}^2+p_m^2)}$;\\
        $L_o \leftarrow L_o+\beta\cdot w_l\cdot {\rm LightEval}(light, \textbf{n}_{lt}, \omega_{lt}) $;
    }
\end{algorithm}

\section{随机自适应光子映射算法}

光子映射算法的核心思想,是通过发射光子+区域查找来近似场景中的间接光照分布。
由于从头开始介绍该类算法需要花费大量功夫,因此这里只对最终实现的随机自适应光子映射算法(SPPM算法)进行简单陈述。

首先,我们需要对直接光照和间接光照两个概念进行精确的定义。
在本章中,直接光照指从光源出发,只经过镜面材质(或只在第一次经过漫反射材质)从而达到相机的光照(路径表达式为$LDS^*C$或$LS^*C$),
而间接光照则对应其余的所有光照(路径表达式为$L(S|D)^+D(S|D)^*C$)。
对于直接光照,一般采用路径追踪算法直接求得,间接光照则通过光子进行计算。

SPPM算法是一个渐进算法,因此需要进行多轮运算才能达到收敛。
每一轮的运算又具体分为三步:
第一步被称为光子通道(Photon Pass),由光源生成大量光子,并进行追踪直到达到一定层数为止;
第二步将场景中的光子储存在一个按空间划分的数据结构中(例如KD树),该结构则被称为光子图(Photon Map);
最后一步为聚集通道(Gather Pass),即通过路径追踪算法进行渲染,并在光线遇到漫反射材质时,根据相交点周围的光子计算间接光照。
由于第一、三步需要光线追踪功能,因此在本框架中由OpTiX完成;第二步则通过Host程序完成。

\subsection{光子通道}

有了之前的基础,实现光子通道这一步只需要再设计一个Ray Generation程序即可。
同时,还要在LightProperty中加入两个计算函数LightDirSample及LightDirPdf,
从而支持对出射光线进行方向上的采样。

关于光子通道的流程见算法\ref{RGPPPass}。这里需要解释一下$\beta'$的作用:
由于保存的光子信息不包含最后一次采样的结果,因此计算光子能量时需要使用到上一轮迭代生成的$\beta$值。

\begin{algorithm}
    \SetKwInOut{KIN}{输入}
    \SetKwInOut{KDATA}{输出}
    \caption{光子通道Ray Generation Program}
    \label{RGPPPass}
    \KIN{索引$index_{x,y}$}
    \KDATA{光子数组${\rm pmap}$}

    获取光源属性$light$(PaddingLightProperty);\\
    根据光源的种类,取出对应的五个计算函数;\\
    $(\textbf{x}_{lt}, \textbf{n}_{lt}) \leftarrow {\rm LightSample}(light)$; \\
    $\omega_{lt} \leftarrow {\rm LightDirSample}(light, \textbf{n}_{lt})$; \\
    $p_{lt} \leftarrow {\rm LightPdf}(light, \textbf{x}_{lt})\cdot {\rm LightDirPdf}(light, \textbf{n}_{lt}, \omega_{lt})$; \\
    $\beta, \beta' \leftarrow 1$;\\    
    $d \leftarrow 0$;\\
    \For{$d < d_{photon\_max}$}
    {
        $(\textbf{x}_{lt}, \omega_{lt}, \beta, \_) \leftarrow {\rm rtTrace}(\textbf{x}_{lt}, \omega_{lt}, \beta, \_)$;\\
        \If{Closest Hit Program中的物体材质是漫反射材质}
        {    
            ${\rm append}\Big({\rm pmap}, (\textbf{x}_{lt}, \omega_{lt}, \beta' / p_{lt})\Big)$;\\
            $d \leftarrow d+1$;
        }
        $\beta' = \beta$。
    }

\end{algorithm}

\subsection{维护光子图}
上一步(光子通道)的输出实际是一个由诸多光子信息组成的数组,接下来则是要将这些光子组织成KD树的结构。
由于这部分内容完全是由Host程序实现,且KD树的搭建属于经典算法,因此这里不再对算法流程进行介绍。

唯一要进行说明的是,这步最终得到的光子图是一个通过单一数组表示的二叉树,
这样的结构有便于后续的GPU程序进行查找。

\subsection{聚集通道}

SPPM算法的聚集部分可以概括为以下几个公式:
\begin{align}
    N_{i+1} &= N_{i}+\gamma M_i \label{SPPM1}\\
    r_{i+1} &= r_i\sqrt{\frac{N_{i+1}}{N_i+M_i}} \label{SPPM2}\\
    \tau_{i+1} &= (\tau_i+\Phi_i)\frac{r_{i+1}^2}{r_i^2} \label{SPPM3}\\
    \Phi_i &= \sum_{j = 1}^{M_i}\beta_jf(\textbf{x},\omega_o, \omega_j) \\
    L_i &= \frac{\tau_i}{N_{sum}\cdot \pi r_i^2}  
\end{align}
其中$r_i$代表搜索半径,$N_i$和$N_{sum}$分别代表总共找到及总共发出的光子数,
$M_i$代表本轮迭代在半径$r_i$中搜索到的光子总数,$\tau_i$代表所有光子带来的光通量总和,
而$L_i$代表最终求得的间接光照强度。

这一部分的实现则基于对\ref{cha:pathtrace}中的两个程序的修改。
对于Closest Hit Program来说,只需在返回值中添加当前材质的属性(MaterialProperty)即可;
而Ray Generation Program的改动则相对更多,算法\ref{RGPGather}描述了该函数的大致流程。

\begin{algorithm}
    \SetKwInOut{KIN}{输入}
    \SetKwInOut{KDATA}{数据}
    \SetKwInOut{KOUT}{输出}
    \caption{聚集通道Ray Generation Program}
    \label{RGPGather}
    \KIN{索引$index_{x,y}$,已发射的光子总数$N_{sum}$}
    \KDATA{累计平均光照亮度$L_{acc}$,总采样数$S$,搜索半径$r_i$,收到光子数$N_i$,光子光通量总和$\tau_i$}
    \KOUT{光照亮度$L_{out}$}
    生成初始光线$(\textbf{x}, \omega_o)$,设定初始值$\beta,L_o,d$;\\
    $\beta' \leftarrow 1$;\\
    \For{$d < d_{max}$}
    {
        $(\textbf{x}, \omega_o, \beta, L_o, mat) \leftarrow {\rm rtTrace}(\textbf{x}, \omega_o, \beta, L_o)$;\\
        \If{$mat$是漫反射材质}
        {
            取得$mat$的Eval计算函数;\\
            $M_i \leftarrow 0$;\\
            $\Phi_i \leftarrow 0$;\\
            \For{$\textbf{x}$周围$r_i$半径内的光子$(\textbf{x}_{ph}, \omega_{ph}, L_{ph})$}
            {
                $\Phi_i \leftarrow \Phi_i + \beta' \cdot L_{ph}\cdot {\rm Eval}(mat, \textbf{x}, \omega_o, \omega_{ph})$\\
                $M_i \leftarrow M_i+1$;
            }
            按照式\ref{SPPM1},式\ref{SPPM2},式\ref{SPPM3}更新$r_i, N_i,\tau_i$;\\
            \textbf{break}\\
        }
        $d \leftarrow d+1$; \\
        $\beta' \leftarrow \beta$;\\
    }
    $L_{acc} \leftarrow (L_o+L_{acc}\cdot S)/(S+1)$;\\
    $L_{out} \leftarrow L_{acc}+\tau_i/(N_{sum}\cdot \pi r_i^2)$;\\
    $S \leftarrow S+1$;
\end{algorithm}

至此,关于SPPM算法的GPU实现已基本介绍完毕。

\section{渲染优化}

除去以上的两个渲染算法外,本人还实现了两个不同的渲染优化算法,这里也进行一些简单说明。

\subsection{基于深度学习的图像去噪}

基于之前提到的工作\cite{NvidiaDenoiser},新版的OpTiX库加入了深度学习的去噪功能,
本人则将其进一步封装成了一个PostProcessor类(名为DLDenoiser)。
由于该去噪功能对应的模块在设计上与OpTiX的光线渲染模块基本互相独立,且只要调用几个简单的API函数就可以使用,
因此这里不再介绍其具体的使用流程。
要注意的一点是,该去噪器的输入除去渲染生成的RGB结果外,还包括相交点位置、相交点向量、Albedo等辅助信息,
因此还需要修改Renderer使其输出些数据。

\subsection{基于方差估计的自适应渲染算法}

\label{adaptive}

该算法源自于2009年Holger Dammertz等人的工作\cite{AdaptiveSampling},主要作用与路径追踪算法上。
大致来说,在每一轮路径追踪结束后,
它该算法会将所有当前帧的的渲染结果按照奇偶分成两部分,然后累加求差,从而估计图像上各个像素点的亮度方差大小。
对于图像中的某一个像素块(这些块具体如何生成详见原文),如果其内部的像素方差平均值达到足够小,该块便不再继续进行采样。

该算法的实现基本在Host程序内完成,故这里不再介绍。涉及到Optix程序的修改主要有两点:
第一点是Ray Generation Program将获得一个包含终止信息的Buffer作为输入,以此来决定当前像素是否需要继续渲染;
第二点则是Ray Generation Program需要额外输出奇数帧下的累计光照强度(偶数帧结果可以通过与原结果相减得到)。

%A Hierarchical Automatic Stopping Condition forMonte Carlo Global Illumination