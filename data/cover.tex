\thusetup{
  %******************************
  % 注意:
  %   1. 配置里面不要出现空行
  %   2. 不需要的配置信息可以删除
  %******************************
  %
  %=====
  % 秘级
  %=====
  secretlevel={秘密},
  secretyear={10},
  %
  %=========
  % 中文信息
  %=========
  ctitle={基于GPU的实验室渲染平台},
  cdegree={工学学士},
  cdepartment={计算机科学与技术系},
  cmajor={计算机科学与技术},
  cauthor={匡正非},
  csupervisor={徐昆教授},
  cassosupervisor={}, % 副指导老师
  ccosupervisor={}, % 联合指导老师
  % 日期自动使用当前时间,若需指定按如下方式修改:
  % cdate={超新星纪元},
  %
  % 博士后专有部分
  catalognumber     = {分类号},  % 可以留空
  udc               = {UDC},  % 可以留空
  id                = {编号},  % 可以留空: id={},
  cfirstdiscipline  = {计算机科学与技术},  % 流动站(一级学科)名称
  cseconddiscipline = {系统结构},  % 专 业(二级学科)名称
  postdoctordate    = {2009 年 7 月——2011 年 7 月},  % 工作完成日期
  postdocstartdate  = {2009 年 7 月 1 日},  % 研究工作起始时间
  postdocenddate    = {2011 年 7 月 1 日},  % 研究工作期满时间
  %
  %=========
  % 英文信息
  %========= 
  etitle={An Introduction to \LaTeX{} Thesis Template of Tsinghua University v\version},
  % 这块比较复杂,需要分情况讨论:
  % 1. 学术型硕士
  %    edegree:必须为Master of Arts或Master of Science(注意大小写)
  %             “哲学、文学、历史学、法学、教育学、艺术学门类,公共管理学科
  %              填写Master of Arts,其它填写Master of Science”
  %    emajor:“获得一级学科授权的学科填写一级学科名称,其它填写二级学科名称”
  % 2. 专业型硕士
  %    edegree:“填写专业学位英文名称全称”
  %    emajor:“工程硕士填写工程领域,其它专业学位不填写此项”
  % 3. 学术型博士
  %    edegree:Doctor of Philosophy(注意大小写)
  %    emajor:“获得一级学科授权的学科填写一级学科名称,其它填写二级学科名称”
  % 4. 专业型博士
  %    edegree:“填写专业学位英文名称全称”
  %    emajor:不填写此项
  edegree={Bachelor of Engineering},
  emajor={Computer Science and Technology},
  eauthor={Kuang Zhengfei},
  esupervisor={Professor Xun Kun},
  eassosupervisor={},
  % 日期自动生成,若需指定按如下方式修改:
  % edate={December, 2005},
  %
  % 关键词用“英文逗号”分割
  ckeywords={图形学,渲染,图形处理器,机器学习},
  ekeywords={Computer Graphics, Rendering, Graphics Processing Unit, Maching Learning}
}

% 定义中英文摘要和关键字
\begin{cabstract}
  利用计算机进行真实化绘制(也称渲染)长期以来一直是计算机图形学领域中的核心技术,在娱乐、医疗、交通等多个行业中都扮演着极为重要的角色。
  图形处理器(GPU)最初是为加速基于光栅化的渲染而产生的硬件,而随着近年来其计算能力的逐步提高,也逐渐被用于基于光线追踪的渲染当中。

  本文通过已有的GPU光线追踪库OpTiX,设计了一套高效、可复用、扩展性强的GPU渲染架构,并将其与传统CPU渲染器进行对照实验。
  此外,利用此架构搭建的平台,本文进一步对基于机器学习的渲染优化算法,及VRay Material展开了研究。

  本文的创新点主要有:
  \begin{itemize}
    \item 设计了一套完整的GPU渲染平台框架;
    \item 在该框架下实现了多个主流渲染算法及优化算法的GPU版本;
    \item 在该框架下模拟了当前最先进的渲染材质之一:VRay Material。
  \end{itemize}

\end{cabstract}

% 如果习惯关键字跟在摘要文字后面,可以用直接命令来设置,如下:
% \ckeywords{\TeX, \LaTeX, CJK, 模板, 论文}

\begin{eabstract}
   Drawing realistic images from geometric models with computer (i.e. Rendering) has been one of the most essential technology in Computer Graphics for a long period, 
   playing important roles in multiple industries including entertainment, medication and transportation, etc.
   A main issue of rendering that (---) most is how to facilitate the rendering process, reaching accurate results with an affordable time cost. 
   Graphics Processing Unit (i.e. GPU), 
\end{eabstract}

% \ekeywords{\TeX, \LaTeX, CJK, template, thesis}
