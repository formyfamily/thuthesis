\thusetup{
  %******************************
  % 注意:
  %   1. 配置里面不要出现空行
  %   2. 不需要的配置信息可以删除
  %******************************
  %
  %=====
  % 秘级
  %=====
  secretlevel={秘密},
  secretyear={10},
  %
  %=========
  % 中文信息
  %=========
  ctitle={基于GPU的实时光线追踪渲染平台},
  cdegree={工学学士},
  cdepartment={计算机科学与技术系},
  cmajor={计算机科学与技术},
  cauthor={匡正非},
  csupervisor={徐昆副教授},
  cassosupervisor={}, % 副指导老师
  ccosupervisor={}, % 联合指导老师
  % 日期自动使用当前时间,若需指定按如下方式修改:
  % cdate={超新星纪元},
  %
  % 博士后专有部分
  % catalognumber     = {分类号},  % 可以留空
  % udc               = {UDC},  % 可以留空
  % id                = {编号},  % 可以留空: id={},
  % cfirstdiscipline  = {计算机科学与技术},  % 流动站(一级学科)名称
  % cseconddiscipline = {系统结构},  % 专 业(二级学科)名称
  % postdoctordate    = {2009 年 7 月——2011 年 7 月},  % 工作完成日期
  % postdocstartdate  = {2009 年 7 月 1 日},  % 研究工作起始时间
  % postdocenddate    = {2011 年 7 月 1 日},  % 研究工作期满时间
  %
  %=========
  % 英文信息
  %========= 
  etitle={An Introduction to \LaTeX{} Thesis Template of Tsinghua University v\version},
  % 这块比较复杂,需要分情况讨论:
  % 1. 学术型硕士
  %    edegree:必须为Master of Arts或Master of Science(注意大小写)
  %             “哲学、文学、历史学、法学、教育学、艺术学门类,公共管理学科
  %              填写Master of Arts,其它填写Master of Science”
  %    emajor:“获得一级学科授权的学科填写一级学科名称,其它填写二级学科名称”
  % 2. 专业型硕士
  %    edegree:“填写专业学位英文名称全称”
  %    emajor:“工程硕士填写工程领域,其它专业学位不填写此项”
  % 3. 学术型博士
  %    edegree:Doctor of Philosophy(注意大小写)
  %    emajor:“获得一级学科授权的学科填写一级学科名称,其它填写二级学科名称”
  % 4. 专业型博士
  %    edegree:“填写专业学位英文名称全称”
  %    emajor:不填写此项
  edegree={Bachelor of Engineering},
  emajor={Computer Science and Technology},
  eauthor={Kuang Zhengfei},
  esupervisor={Professor Xun Kun},
  eassosupervisor={},
  % 日期自动生成,若需指定按如下方式修改:
  % edate={December, 2005},
  %
  % 关键词用“英文逗号”分割
  ckeywords={图形学,渲染,图形处理器,机器学习},
  ekeywords={Computer Graphics, Rendering, Graphics Processing Unit, Maching Learning}
}

% 定义中英文摘要和关键字
\begin{cabstract}
  利用计算机进行真实化绘制(也称渲染)长期以来一直是计算机图形学领域中的核心技术,在娱乐、医疗、交通等多个行业中都扮演着极为重要的角色。
  图形处理器(GPU)最初是为加速基于光栅化的渲染而产生的硬件,而随着近年来其计算能力的逐步提高,也逐渐被用于基于光线追踪的渲染当中。

  本文通过已有的GPU光线追踪库OpTiX,设计了一套高效、可复用、扩展性强的GPU渲染架构。
  利用此架构搭建的平台,本文进一步将一些主流的渲染算法、渲染优化算法、
  以及时下热门的VRay材质移植至GPU下,并对最终的实现效果进行了一些测试。

  本文的创新点主要有:
  \begin{itemize}
    \item 设计了一套完整的GPU渲染平台框架;
    \item 设计了多个主流渲染算法及优化算法的GPU版本;
    \item 研究并模拟了当前最先进的渲染材质之一:VRay Material。
  \end{itemize}

\end{cabstract}

% 如果习惯关键字跟在摘要文字后面,可以用直接命令来设置,如下:
% \ckeywords{\TeX, \LaTeX, CJK, 模板, 论文}

\begin{eabstract}

  The use of computers for realistic rendering (also known as rendering) has long been a core technology in the field of computer graphics, playing an extremely important role in many industries such as entertainment, medicine, and transportation.
  The graphics processing unit (GPU) was originally developed to accelerate raster-based rendering, and with the increase in it's power in recent years, it has gradually been used in ray-tracing-based rendering too.

  This paper designs an efficient, reusable and extensible GPU rendering framework through the existing GPU ray tracing library OpTiX.
  Using the program built with this framework, this article will further port some mainstream rendering algorithms, rendering optimization algorithms,
  and the VRay materials to the GPU, and test the results of implementation by several evaluations.

  The main innovations of this paper are:
  \begin{itemize}
    \item Designed a complete GPU rendering platform framework;
    \item Designed GPU versions of multiple mainstream rendering algorithms and optimization algorithms;
    \item Simulated one of the most advanced rendering materials available today: VRay Material.
  \end{itemize} 
\end{eabstract}

% \ekeywords{\TeX, \LaTeX, CJK, template, thesis}
