\chapter{VRay材质模拟}
\label{cha:vray}

在第四章中,我们用将材质的功能归结成了三个计算函数(Sample、Pdf以及Eval函数),
但对于它们内部的实现并没有展开详细的说明。事实上,对于不同的材质而言,
这些函数往往会有着极大的差异性,而它们对渲染的整体效率、收敛速度以及最终结果都起着至关重要的效果。
本章将会对该渲染器中最为复杂的一个材质——VRay模拟材质的具体情况进行介绍。

VRay材质\cite{VRay}是目前渲染工业界最为著名的材质之一,它支持的功能十分广泛,
渲染效果出众,被运用在诸多主流的渲染软件(如3ds Max、Maya等)上。
严格意义上说,VRay材质实际上由二十多种不同的子材质组成,
本文只对其中的VRayMtl和VRayMtl2Sided两种材质进行了模拟。


\section{VRayMtl材质}

根据官方对VRayMtl的介绍,这种材质共由3层(Layer)组成:
漫反射层(Diffuse)、反射层(Reflection)、折射层(Refraction)。
为了满足能量守恒定律,这里规定当光线达到该材质的表面时,以上三层所分得的能量之和应当不大于入射光的能量。
具体的分配方法是,先由反射层从入射光吸收部分能量,然后由反射层吸收,最后再是漫反射层。
换言之,以上三层在分配能量上的的优先度顺序为:反射层>折射层>漫反射层。
各层吸收能量的多少实际是由该层的颜色值所决定的,但对于反射层来说,还会加入一个菲涅尔效应的影响。

下面要介绍的内容便是本人通过对VRayMtl材质进行大量观察、测试后得出的模拟材质。
首先,我会分别介绍上述三个层的模拟情况;其次介绍三层的权重应当如何设置;再接着会介绍对半透明层的分析;
最后是材质的采样函数以及概率密度函数。

\subsection{漫反射层}

在VRayMtl中,漫反射层是三层中最为简单的部分。
它一共由包含两个参数:材质颜色$c_{d}$以及粗糙度系数$roughness$。
经过分析,本人在这里采用了Lambert模型与Phong-Blinn模型的混合模型来进行模拟:
\begin{equation}
f_d(\omega_o, \omega_i) = \frac{0.5}{\pi} + 0.5C_{pb}\cdot c_{dif} \cdot \text{dot}(\omega_i, \omega_h)^{K_s(1-roughness)}
\end{equation}
其中$C_{pb}$为Phone-Blinn模型的归一化系数(该系数拥有解析解\cite{Phong}),$K_s$则为光泽度系数,在实验中被设置为(TODO:0.5)。
可以看出,当$roughness=1$时(此时$C_{pb}=\frac{1}{\pi}$),该模型实际相当于Lambert漫反射模型。
% http://www.farbrausch.de/~fg/articles/phong.pdf

\subsection{反射层}

VRayMtl材质的反射层共拥有十余个参数,而在本人的模拟中所采用的有:
反射颜色$c_{r}$,反射光泽度$g_r$,
反射层折射系数$\eta_{r}$(该系数仅仅用于计算菲涅尔系数,通常与折射层的折射系数保持一致),
以及最大反射深度$d_{maxr}$。

反射层的BRDF直接采用了目前最先进的BRDF模型之一Microfacet GTR(又称GGX)模型\cite{GTR},即:
\begin{equation}
f_r(\omega_o, \omega_i) = c_r \cdot \frac{D(\omega_h)G(\omega_i,\omega_o)F_r(\omega_i)}{4\cos\theta_o\cos\theta_i} 
\end{equation}

其中$\omega_h=\omega_o+\omega_i$,$D$为法向量分布函数,$G$为遮挡系数,$F_r$则为菲涅尔函数(这里采用Schlick近似函数)。
在GGX中,$D$函数和$F_r$函数分别为:
\begin{align}
D(\omega_h) &= \frac{\alpha^2}{\Big(\alpha^2\cos^2\theta_h + \sin^2\theta_h\Big)^2}\\
F_r(\omega_h) &= \Big(\frac{1-\eta_r}{1+\eta_r}\Big)^2+\Big(1-\Big(\frac{1-\eta_r}{1+\eta_r}\Big)^2\Big)(1-\cos\theta_h)^5
\end{align}
其中,$\alpha$对应的是材质的粗糙度系数,需要通过$g_r$来确定。在实现中,该值最终被设置为$\sqrt{1-g_r}$。
另外,$G$函数由于格式较为复杂,且不涉及其他新的参数,所以这里不再对其展示。以上就是反射层BRDF的计算方法。

最后,还需要说明一下$d_{maxr}$的用途。
在VRay当中有一点看似不可思议的设定,即对于一条入射光线而言,如果它经过的反射层数量超过了当前反射层的$d_{maxr}$值,
那么该光线便会直接将该反射层穿透(相当于打在一个折射系数为1的折射层上)。
这样设计的目的主要是为了防止光线被困在复杂的场景之中,不断反射直到超过最大深度而消失,
从而无法取得任何光照。尽管穿透的设定可能会导致渲染结果失真,但这样处理确实能够更好地保持住能量守恒的性质。

\subsection{折射层}

和反射层类似,折射层采用的参数分别有:
折射颜色$c_t$,折射光泽度$g_t$,
折射系数$\eta$,以及最大折射深度$d_{maxt}$。

折射层的BTDF函数同样采用的是基于Microfacet的BTDF\cite{MicrofacetBTDF}(见下式),且$D$函数、$G$函数和反射层保持一致。
唯一不同的点在于,折射层中的粗糙度系数$\alpha=(1-g_t)$。
\begin{equation}
    f_t(\omega_o, \omega_i) = c_t \cdot \frac{\eta^2 D(\omega_h)G(\omega_i,\omega_o)(1-F_r(\omega_o))}{((\omega_o\cdot\omega_h) + \eta (\omega_i\cdot\omega_h))^2}\cdot \frac{|\omega_i\cdot\omega_h||\omega_o\cdot\omega_h|}{\cos\theta_o\cos\theta_i} 
\end{equation}
其中$\omega_h=\omega_o+\eta\omega_i$。

除了上面的内容之外,在折射层中还需要考虑全反射的情况。当$\eta<1$时,从$\omega_i$射入的光线可能无法被折射到介质的另一端,
这时候本模型便会按照反射层的模型进行计算(但仍然采用折射层的参数)。需要注意的是,全反射仍然被视作与折射层之间的交互,因此最终会增加光线的折射次数(而非反射次数)。

另外,VRay对于折射层还增加了衰减效果(Fog),即一些介质中的光线会在前进过程中因为散射而逐渐减弱的情况。
本人使用了其中的两个参数:衰减颜色$c_f$和衰减系数$\sigma_f$,对这一效果也进行了模拟。

具体的做法需要借助于OpTiX的帮助。在Closest Hit Program中,
OpTiX会提供当前光线在两次相交点之间的运行长度$t$,这对于我们实现该效果至关重要。
当光线是从物体的内部向外射出时(即$\omega_o\cdot \textbf{n} < 0$时),
便需要考虑它在介质内部的衰减。这里采用的是普通的指数衰减模型:
\begin{equation}
   L_i(\textbf{x}+t\omega, \omega) = e^{-\sigma_f t c_f}\cdot L_o(\textbf{x}, \omega) 
\end{equation}

\subsection{权重分配方式}

至此三个分层内部的模型都已经定义完毕,接下来则需要考虑如何对它们进行权重分配。
设反射层、折射层以及漫反射层对应的权重分别为$w_r, w_t, w_d$,
根据VRay文档中提供的说明以及VRayMtl的运行效果,本模型最后采用的分配方式为:
\begin{equation}
w_r = c_rF_r(\omega_o) ; \quad w_t = c_t(1-w_r) ; \quad w_d = c_d(1-w_r-w_t).
\end{equation}

注意这些权重值包含了材质各层的颜色信息。
接着,我们只需要将三层BSDF函数中的颜色值替换成上述权重,
然后将三份结果进行累加便可以得到最终的输出值了。

\subsection{采样函数}
\label{VRaySample}
首先来看各层的采样方式:对于漫反射层,我们可以直接在半球面上进行均匀采样;
而对于反射层、折射层,则可以参考GGX模型中的做法,利用$D$函数进行采样。

VRayMtl模拟模型由以上三层混合而成,其采样过程分为两步:先在三者中按照某种概率分布$p_{l}$随机选择一层,然后利用该层的材质来进行采样。
另一方面,模型的采样概率密度函数$p$则通过将各层的概率密度函数乘上其被选中的概率,再进行累加而获得。即:
\begin{equation}
p(\omega_o) = P(\text{reflect})p_r(\omega_o)+P(\text{refract})p_t(\omega_o)+P(\text{diffuse})p_d(\omega_o);
\end{equation}
各层被选中的概率则和上一章中的权重直接相关:
\begin{equation}
    P(\text{reflect}) = |w_r|; \quad P(\text{refract}) = |w_t|; \quad P(\text{diffuse}) = 1-|w_r|-|w_t|.
\end{equation}
    
\section{VRayMtl2Side材质}

VRayMtl2Side材质是通过两种VRay材质组合而成的,这两个材质分别分布在物体的正反两面上。
该材质提供了一个透明度参数$tr$,通过该参数对两种材质进行叠加。

根据实际运行的情况,本人将这种材质的BSDF模拟函数定义为:
\begin{equation}
    f(\omega_i, \omega_o) = 
    \begin{cases}
    (1-tr)\cdot f_{front}(\omega_i, \omega_o)+tr\cdot f_{back}(-\omega_i, -\omega_o)& \omega_o \cdot \textbf{n} > 0\\
    tr\cdot f_{front}(\omega_i, \omega_o)+(1-tr)\cdot f_{back}(-\omega_i, -\omega_o)& \omega_o \cdot \textbf{n} < 0\\
    \end{cases}
\end{equation}
其中$f_{front},f_{back}$分别代表正反面两种材质的BSDF函数。其采样和概率密度函数的计算和\ref{VRaySample}小节中的过程类似。

在程序实现中,VRayMtl2Side材质的MaterialProperty包含两种下属材质的ID以及透明度参数。
在它的计算函数中,首先会根据ID找到下属材质的MaterialProperty,然后再调用它们的计算函数从而求得结果。

