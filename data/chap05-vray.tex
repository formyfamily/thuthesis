\chapter{VRay材质模拟}
\label{cha:vray}

在第四章中,我们将与材质相关的计算总结成了三个计算函数(Sample、Pdf以及Eval函数),
但对于它们内部的实现并没有展开详细的说明。事实上,对于不同的材质而言,
这些函数的内部往往会有着极大的差异性,而它们的实现情况对渲染的整体效率、收敛速度以及最终结果都起着至关重要的效果。
这一章将会对本渲染器中已经实现的其中一个材质——VRay模拟材质的具体实现情况进行介绍。

VRay材质\footnote{https://docs.chaosgroup.com/display/VRAY3MAX/Materials}是目前渲染工业界最为著名的材质之一,它支持的功能十分广泛,
渲染效果出众,被运用在诸多主流的渲染软件(如3ds Max、Maya等)上。
严格意义上说,VRay材质实际上由二十多种不同的子材质组成,
本渲染器只对其中的VRayMtl和VRayMtl2Sided两种子材质进行了模拟。


\section{VRayMtl材质}

根据官方对VRayMtl的介绍,这种材质共由3层(Layer)组成:
漫反射层(Diffuse)、反射层(Reflection)、折射层(Refraction)。
为了满足能量守恒定律,该材质规定当光线到达其表面时,以上三层所分得的能量之和应当不大于入射光的能量。
具体的分配方法是,先由反射层从入射光吸收部分能量,然后再由折射层吸收一部分,最后再由漫反射层吸收。
换言之,以上三层在分配能量上的的优先度顺序为:反射层>折射层>漫反射层。
各层吸收能量的多少实际由该层的颜色值所决定,但对于反射层来说,还会考虑菲涅尔效应的影响。

下面要介绍的内容便是本人通过对VRayMtl材质进行大量观察、测试后得出的模拟材质。
首先,我会分别介绍上述三个层的模拟情况;其次介绍三层的权重应当如何设置;
最后是材质的采样函数以及概率密度函数。

\subsection{漫反射层}

在VRayMtl中,漫反射层是三层中相对而言最为简单的部分。
它一共包含了两个参数:材质颜色$c_{d}$以及粗糙度系数$roughness$。
经过分析,本人在这里采用了Lambert模型与Phong-Blinn模型的混合模型来进行模拟:
\begin{equation}
f_d(\omega_o, \omega_i) = \frac{0.5}{\pi} + 0.5C_{pb}\cdot c_{dif} \cdot \text{dot}(\omega_i, \omega_h)^{K_s(1-roughness)}
\end{equation}
其中$C_{pb}$为Phone-Blinn模型的归一化系数(该系数拥有解析解\cite{Phong}),$K_s$则为光泽度系数,在实验中被设置为0.8。
可以看出,当$roughness=1$时(此时$C_{pb}=\frac{1}{\pi}$),该模型实际相当于Lambert漫反射模型。
% http://www.farbrausch.de/~fg/articles/phong.pdf

\subsection{反射层}

VRayMtl材质的反射层共拥有十余个参数,在本人的模拟中采用的有:
反射颜色$c_{r}$,反射光泽度$g_r$,
反射层折射系数$\eta_{r}$(该系数仅仅用于计算菲涅尔系数,且通常与折射层的折射系数保持一致),
以及最大反射深度$d_{maxr}$。

反射层的BRDF直接采用了目前最先进的BRDF模型之一GGX Microfacet模型\cite{Walter2007MicrofacetMF}\cite{GTR},即:
\begin{equation}
f_r(\omega_o, \omega_i) = c_r \cdot \frac{D(\omega_h)G(\omega_i,\omega_o)F_r(\omega_i)}{4\cos\theta_o\cos\theta_i} 
\end{equation}

其中$\omega_h=\omega_o+\omega_i$,$D$为法向量分布函数,$G$为遮蔽函数,$F_r$则为菲涅尔函数(这里采用Schlick近似函数)。
$D$函数和$F_r$函数的定义分别为:
\begin{align}
D(\omega_h) &= \frac{\alpha^2}{\Big(\alpha^2\cos^2\theta_h + \sin^2\theta_h\Big)^2}\\
F_r(\omega_h) &= \Big(\frac{1-\eta_r}{1+\eta_r}\Big)^2+\Big(1-\Big(\frac{1-\eta_r}{1+\eta_r}\Big)^2\Big)(1-\cos\theta_h)^5
\end{align}
其中,$\alpha$表示的是材质的粗糙度系数,需要通过$g_r$参数来确定。在实现中,该值最终被设置为了$\sqrt{1-g_r}$。
$G$函数由于格式较为复杂,且不涉及到其他新的参数,所以这里不进行展示。以上就是反射层BRDF的计算公式。

最后,在这里还需要说明一下$d_{maxr}$的用途。
在VRay当中有一条看似不可思议的设定,即对于一条入射光线而言,如果它之前已经经过的反射层数量超过了当前反射层的$d_{maxr}$值,
那么该光线便会直接将该反射层穿透(相当于打在一个折射系数为1的折射层上)。
这样设计的目的主要是为了防止光线被困在复杂的场景之中,不断反射直到超过迭代最大深度消失,
从而无法取得任何光照。尽管穿透的设定很可能会导致渲染结果失真,
但这样处理确实能使得渲染结果更好地保持住能量守恒的性质。

\subsection{折射层}

和反射层类似,折射层采用的参数分别有:
折射颜色$c_t$,折射光泽度$g_t$,
折射系数$\eta$,以及最大折射深度$d_{maxt}$。

折射层的BTDF函数同样采用的是基于Microfacet的BTDF\cite{Walter2007MicrofacetMF}(见式\ref{VRayBTDF}),且$D$函数、$G$函数和反射层保持一致。
唯一不同的点在于,折射层中的粗糙度系数$\alpha=(1-g_t)$。
\begin{equation}
    \label{VRayBTDF}
    f_t(\omega_o, \omega_i) = c_t \cdot \frac{\eta^2 D(\omega_h)G(\omega_i,\omega_o)(1-F_r(\omega_o))}{((\omega_o\cdot\omega_h) + \eta (\omega_i\cdot\omega_h))^2}\cdot \frac{|\omega_i\cdot\omega_h||\omega_o\cdot\omega_h|}{\cos\theta_o\cos\theta_i} 
\end{equation}
其中$\omega_h=\omega_o+\eta\omega_i$。

折射层中还需要考虑全反射的情况。当$\eta<1$时,从$\omega_o$射入的光线可能无法被折射到介质的另一端,
这时候本模型便会按照反射层的模型进行计算(但仍然采用折射层的参数)。
需要注意的是,全反射仍然被视作与折射层之间的交互,因此最终会增加光线的折射次数(而非反射)。

另外,VRay还在折射层中增加了衰减效果(Fog),即一些介质中的光线会在前进过程中因为散射而逐渐减弱的情况。
对于这一部分,本人使用了其中的两个参数:衰减颜色$c_f$和衰减系数$\sigma_f$,对该效果也进行了模拟。

这里具体的做法需要借助于OpTiX的帮助。在Closest Hit Program中,
OpTiX会提供当前光线从起点到当前相交点之间的前进距离$t$,这对于我们实现该效果至关重要。
当一条光线是从物体的内部向外射出时(即$\omega_o\cdot \textbf{n} < 0$时),
便需要考虑它在介质内部的衰减情况。这里采用了一般的指数衰减模型进行计算,即:
\begin{equation}
   L_i(\textbf{x}+t\omega, \omega) = e^{-\sigma_f t c_f}\cdot L_o(\textbf{x}, \omega) 
\end{equation}

\subsection{权重分配方式}

至此三个分层各自的模型都已经定义完毕,接下来则需要考虑如何对它们进行权重分配。
设反射层、折射层以及漫反射层对应的权重分别为$w_r, w_t, w_d$,
根据VRay文档中提供的说明以及VRayMtl的运行效果,本模型最后采用的分配方式为:
\begin{equation}
w_r = c_rF_r(\omega_o) ; \quad w_t = c_t(1-w_r) ; \quad w_d = c_d(1-w_r-w_t).
\end{equation}

注意这些权重值包含了材质各层的颜色信息。
接着,我们只需要将三层BSDF函数中的颜色值替换成上述权重,
然后将三份模型的计算结果进行累加便可以得到最终的输出值了。

\subsection{采样函数}
\label{VRaySample}
首先来看各个分层是如何完成采样的:对于漫反射层,这里的选择是在半球面上均匀采样;
而对于反射层、折射层,则参考GGX模型中的做法,利用$D$函数来进行采样。

VRayMtl模拟模型由以上三层混合而成,因此其采样过程分为两步:先在三者中按照某种概率分布$P$随机选择一层,然后利用该层的采样方式进行采样。
对于模拟模型的采样概率密度函数$p$,则通过将各层的概率密度函数乘上其被选中的概率再进行累加而获得。即:
\begin{equation}
p(\omega_o) = P(\text{reflect})p_r(\omega_o)+P(\text{refract})p_t(\omega_o)+P(\text{diffuse})p_d(\omega_o);
\end{equation}
各层被选中的概率和上一章中计算得到的权重直接相关:
\begin{equation}
    P(\text{reflect}) = |w_r|; \quad P(\text{refract}) = |w_t|; \quad P(\text{diffuse}) = 1-|w_r|-|w_t|.
\end{equation}
    
至此,关于VRayMtl模拟材质的介绍基本完成。
事实上,VRayMtl材质中还有一些诸如半透明、焦散等高级的效果,
这些效果目前暂时还没有实现。

\section{VRayMtl2Side材质}

VRayMtl2Side材质是通过两个VRay材质组合而成的,这两个材质分别分布在该材质的正反两面上。
这种材质还包含了一个透明度参数$tr$,用来控制最后渲染结果中两种材质的权重大小。

根据实际运行的情况,本人将这种材质的模拟模型定义为:
\begin{equation}
    f(\omega_i, \omega_o) = 
    \begin{cases}
    (1-tr)\cdot f_{front}(\omega_i, \omega_o)+tr\cdot f_{back}(-\omega_i, -\omega_o)& \omega_o \cdot \textbf{n} > 0\\
    tr\cdot f_{front}(\omega_i, \omega_o)+(1-tr)\cdot f_{back}(-\omega_i, -\omega_o)& \omega_o \cdot \textbf{n} < 0\\
    \end{cases}
\end{equation}
其中$f_{front},f_{back}$分别代表正反面两种材质的BSDF函数。
其采样和概率密度函数的计算和\ref{VRaySample}小节中的过程类似。

在程序实现中,VRayMtl2Side材质的内容共包含两个下属材质的ID及透明度参数。
在它的计算函数中,首先会根据ID找到下属材质的\textbf{MaterialProperty},然后再通过调用它们的计算函数来求得结果。

