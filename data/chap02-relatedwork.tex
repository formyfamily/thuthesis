\chapter{相关工作}
\label{cha:relatedwork}
本章将主要对渲染算法的重要技术进行简单的介绍。
由于研究的内容同时涉及光线追踪的并行化技术与部分机器学习相关算法,
因此也会对这些领域的工作进行部分整理。

\section{渲染算法}

正如上一章所提到的,渲染技术从提出至今已经拥有超过50余年的历史,已经是一门相当成熟的学科。

%^ Immel, David S.; Cohen, Michael F.; Greenberg, Donald P. (1986), "A radiosity method for non-diffuse environments" (PDF), Siggraph 1986: 133, doi:10.1145/15922.15901, ISBN 978-0-89791-196-2
%^ Kajiya, James T. (1986), "The rendering equation" (PDF), Siggraph 1986: 143–150, doi:10.1145/15922.15902, ISBN 978-0-89791-196-2
%Watt, Alan; Watt, Mark (1992). "12.2.1 The path tracing solution to the rendering equation". Advanced Animation and Rendering Techniques: Theory and Practice. Addison-Wesley Professional. p. 293. ISBN 978-0-201-54412-1.
%Kajiya, James T.; Von Herzen, Brian P. (1984), "Ray tracing volume densities", Siggraph 1984, 18 (3): 165, CiteSeerX 10.1.1.128.3394, doi:10.1145/964965.808594

渲染技术有着完整而牢固的理论基础。
由David Immel等人\cite{RenderingEquation1},以及James Kajiya\cite{RenderingEquation2}在1986年同时提出的渲染方程,被认为是最著名的渲染理论之一:
该方程利用物理方法近似描述了在几何光学模型中(不考虑光的波长效应)光照辐射度的传输形式,成为了后来大量渲染算法的理论来源。
\begin{equation}
L_o(\mathbf{x}, \omega_o)=L_e(\mathbf{x},\omega_o)+\int_{\Omega} f(\mathbf{x}, \omega_i, \omega_o) L_i(\mathbf{x}, \omega_i) (\omega_i \cdot \mathbf{n}) \operatorname d\omega_i
\end{equation}
1992年Alan Watt等人的工作\cite{Fredholm},进一步得出渲染方程是Fredholm积分中的第二类积分。
除此之外,针对不同种类的渲染需求,许多相对应的理论也先后被提出,
如James Kajiya等人在1984年提出过体渲染方程\cite{VolumnRenderingEquation},为传输介质非真空情况下的体渲染提供了支持;
Adam M Smith等人提出的短暂渲染方程\cite{TransientEquation},用于通过渲染生成Time-Of-Flight数据的计算等等。


写得不好{
    在确定渲染工作的最终目标是求解渲染方程后,许多至今仍在沿用的渲染算法相继出现。
    按照工作原理,这些算法大致可以分为基于光线追踪的渲染算法和基于光栅化的渲染算法。   
    事实上,它们的框架早在渲染方程提出之前就都已经形成,
    然而时至今日,这些算法几乎都以该方程作为其理论基础。
}
下面将对它们进行更加详细的介绍。

\subsection{基于光线追踪的渲染算法}

所谓光线追踪框架,便是通过摄像机向各个方向发射光线,并通过迭代计算
来获得包含物体表面反射、折射的渲染结果。
通常认为,最早提出的光线追踪算法是由Arther Appel在1968年提出的光线投射算法\cite{RayCasting}。
事实上,该算法仅完成了光线追踪流程中的第一步,相机发射的光线在与物体相交之后便直接求得光照值。
真正意义上的光线追踪算法,则是由Turner Whitted于1979年提出的递归式光线追踪算法\cite{WhittedRayTracing}(也被称作Whitted风格的光线追踪算法)。
在该算法中第一次提到了次级光线的概念,及初始光线与物体表面交互后所形成的新的光线。下图是一张由递归式光线追踪算法所生成的经典图像:

以上的算法已经能够支持足够复杂的场景,但是仍然存在一点明显的不足,
即发射的光线与次级光线都有着确定的路径,因此一方面可能会带来严重的锯齿问题,
另一方面无法实现软阴影、运动模糊等所谓“软”效果。
为了改进这点缺陷,1984年由Robert Cook提出的分布式光线追踪算法\cite{DistributiveRayTracing}首次在光线生成上加入了不确定性,
并引入了随机采样的机制,从而良好地解决了以上问题。
%http://artis.inrialpes.fr/Enseignement/TRSA/CookDistributed84.pdf
% figure 

1986年,在渲染方程问世后不久,其提出者James Kajiya很快又提出了路径追踪算法\cite{PathTracing},
这一算法与相比之前增加了几点重要的创新:
首先利用蒙特卡洛积分求解渲染方程的思路提出基于物体BRDF函数的采样方法;
其次将以往路径追踪算法中存储光线的树结构变为了多条单一的路径结构。
由于理论性的完善,路径追踪算法被认为是第一个“无偏”的光线追踪算法,
其实际的运行效果也比以往有了巨大的提升。
% figure

路径追踪算法尽管获得足够精确的计算结果,却在收敛速度上较为缓慢。
为了提升收敛速度,有许多针对路径追踪的优化算法被提出。其中,
一部分算法试图从采样方式上进行优化,如Veach and J.Guibas于1995提出的复合重要性采样\cite{MultipleImportanceSampling}算法,
通过从多个不同的分布进行联合采样的策略,来减小一些场景的误差(效果可见下图);
William J. Morokoff等人在1995年提出拟蒙特卡洛积分法\cite{QuasiMonteCarlo},
通过将利用确定性方法生成的序列代替随机数提高收敛速度,这一思路同样被运用到了路径追踪算法中;
许多工作如([Belcour et.al. 2013], ???)试图通过已有的渲染结果来选择接下来的采样密度分布,这些算法又被统称为适应性采样。
有的算法则直接对路径追踪生成的图片进行过滤,从而起到降噪效果,这其中的工作包含有:
TODO:[Sen and Darabi, 2011]??, [Kalantari et al. 2015]??, [Rousselle et al. 2011]??
等等

相比对算法中的某些步骤进行优化,另一些工作则考虑如何对算法本身做出改进。
Lafortune在1996年提出的双向路径追踪算法\cite{BidirectionalPathTracing}便是典型的代表之一,
其基本思路是在每轮迭代中,同时从摄像机与光源各发射并追踪一条光线,然后对两条路径上的节点两两之间分别计算光照。
这一算法有效地降低了噪声,并使得许多在原算法中难以追踪的路径也被得到计算。

TODO: 梅特波利斯算法

对于如何处理从光源发射并追踪的光线,和双向路径追踪算法不同的另一条思路是,
将这些光线信息直接存储在场景的物体表面上,然后通过统计一定范围内的光线的数量,
来获取表面上任意一点实际光照强度的近似值。这类思路所主导的算法被称为光子映射算法,而上述的光线也被称为光子。
光子映射算法最早的出现在1993年(谁的工作?),而在此之后又受到多次改良。
Toshiya Hachisuka等人于2008年提出的渐进式光子映射算法\cite{ProgressivePhotonMapping},
大大缩小了光子映射所占的内存空间和运行时间复杂度;
仅一年之后,Toshiya Hachisuka等人又在2009年发表了随机渐进式光子映射算法\cite{StochasticProgressivePhotonMapping},
其相比渐进式光子映射加入了随机采样的功能,从而能够支持之前所提到的各类“软”效果。

TODO:顶点链接与合并


\subsection{基于光栅化的渲染算法}

按照严格定义,所谓光栅化其实是指是图形渲染管线中,将输入场景中的几何数据(又名顶点数据)转换为片元数据(从而对应到屏幕中的各个像素点)的过程。
对于大多数采用管线的渲染算法而言,光栅化都是不可或缺的一个关键步骤,因此这一类渲染算法也被称之为光栅化渲染。

TODO:算法理论
%Siggraph 1985 - "Fast Spheres, Shadows, Textures, Transparencies, and Image Enhancements in Pixel-Planes," Henry Fuchs,et. al.
%2000 Sig/Euro Workshop on Graphics Hardware - "Tiled Polygon Traversal Using Half-Plane Edge Functions," Joel McCormack, Robert McNamara
%Siggraph 1988 - "A Parallel Algorithm for Polygon Rasterization,"  Juan Pineda
%Siggraph 2005 - “Resolution-Independent Curve Rendering Using Programmable Graphics Hardware,” Charles Loop, Jim Blinn

和光线追踪算法不同的一点是,光栅化渲染算法不仅仅针对三维场景的渲染,
而同样可以用于二维场景、图形界面等其他需求,具有更加广泛的适应性。
除此之外,光栅化渲染由于拥有通过固定的流水线完成并行计算,
因此速度相比前者快上许多,几乎被当前所有的实时渲染技术所采用。
然而,光栅化算法也有自身的弱势。光栅化算法并没有在渲染中对次级光线进行跟踪,
加之对于实时渲染而言,运行时间上有着严格限制,因此在效果上的表现一般会差于光线追踪算法。
为了弥补这一点缺失,有许多技术试图采用近似的方法来达到更加真实的效果。
这些技术普遍采用的思路是将复杂的光学方程进行拆分,
把所求光照值分解成多个容易计算,或可以通过预处理得到的组成部分,再通过叠加形成最终的图像。
下面将对一些比较典型的工作进行介绍。

(这些与后边的论文内容关系不大,如果有时间的话再考虑添加)

TODO:阴影——环境光遮蔽

TODO:预计算——全局光照、blabla

TODO:体渲染计算

以上所述的这些工作,仅仅只是该领域中的冰山一角。
事实上,由于来自工业界(尤其是娱乐行业为首)的需求之庞大,
基于光栅化的渲染技术(或者实时渲染技术)在很长时间来都属于相当热门的研究课题。
正因如此,该技术的发展速度相当迅速,与之相关的工作浩如烟海,从而也出现了不少专门对算法进行整理介绍的书籍。
其中,较为有名的包括《Real-time Rendering》\cite{RealTimeRendering},《Physically-Based Rendering》\cite{PhysicallyBasedRendering}(又名PBR)
等书,都已经成为该领域中的经典之作,对目前的渲染行业产生了深刻的影响。


\section{光线追踪渲染的并行化}

自从光线追踪框架被提出以来,便有很多学者对其可并行化产生了强烈兴趣。
按照硬件基础的不同,并行化光线追踪技术可以分为两个两种:
第一种是基于CPU的并行技术,主要通过多CPU机器、多核CPU、网络以及分布式系统实现并行;
第二种是基于GPU的并行技术,利用GPU的通用并行计算功能实现实现管线追踪框架。
下面将对两种技术依次进行介绍。

\subsection{基于CPU的并行化}

忽略掉运行机器规模的限制,早期的CPU并行渲染技术在速度上便已经可以达到惊人的水准。
最早被称做实时的光线追踪渲染实际早在1982年出现,由Osaka University带领的团队利用
一台含有514个处理器的大型并行计算机LINKS-1达到了极快的渲染速度\cite{LINKS1}。这台机器规模之大,
直到1984年仍然是世界上能力最强的计算机之一。

尽管这项工作在效果上表现出色,但由于其所用的机器过于庞大,因此很难继续拓展下去。
1986年,Mike Muuss在设计BRL-CAD时另辟蹊径,试图采用网络链接的分布式系统实现并行化,
同样取得了良好的运行效果(速度达到了每秒一帧以上)\cite{BRLCAD}。值得一提的是,BRL-CAD本身作为一款CAD软件,至今仍在开发之中。

更进一步地,人们开始期望实时光线追踪在体型普通的单台机器上运行。于是一方面,
渲染界的学者们开始思考如何在实现上尽可能减少算法的运算量,而另一方面,
CPU的设计者们也在不断设计更加强大的芯片与并行架构,逐渐接近光线追踪算法的运算需求。
直至2008年,Intel公司在一次展示中,利用一款共包含16核的Xeon Tigerton系统完成了720p分辨率下的事实光线追踪,
刷新频率甚至达到了14-29赫兹\cite{QuakeWars}。

然而在此之后,随着新CPU速度的提升逐渐放缓,基于CPU的并行光线追踪技术最终也达到
了发展瓶颈。于是,部分学者们开始将目光转移到原本用于光栅化渲染的图形处理器上,
探求通过另一种硬件来实现这一目标的可能性。

\subsection{基于GPU的并行化} 

早在2002年,Tim Purcell便首次提出了通过GPU实现的全局光照算法,为后来GPU光线追踪的出现埋下了萌芽。
2009年,来自Nvidia公司的Austin Robison展示了首个在GPU上运行的路径追踪渲染器,
几乎同时,Nvidia公司正式向外公布了对后来影响巨大的GPU光线追踪库——OpTiX\cite{Optix}。

TODO:OpTiX的一张示意图

OpTiX的问世归功于不断走向成熟的通用图形处理器(GPGPU)。
相比原本的GPU而言,GPGPU拥有者两点至关重要的特性:
首先,它不存在管线编程的限制,这使得光线追踪里例如采样,计算反射光等复杂操作都能被实现;
其次,也是最为关键的一点,便是在较为成熟的GPGPU中,内核程序也拥有调用函数的能力,因此可以实现递归操作。这是光线追踪算法的核心需求所在。

在此之后的时间内,利用GPU实现实时光线追踪的热度一路上涨,一些商用化的成果也慢慢呈现出来。
AMD公司利用自己研发的GPUOpen Radeon ProRender渲染器,在自家的Vega系列GPU上启用了实时光线追踪功能\cite{Vega};
NVIDIA公司在新发布的GeForce 20系列显卡中,加入了名为RTX的实时光线追踪功能\cite{RTXOn}。
除此之外,越来越多的游戏诸如《战地5》,《我的世界》等,都已经开始实际应用起该项技术。

\section{机器学习与渲染}

机器学习和渲染技术有着密不可分的联系。为渲染而诞生的图形处理器
加速了机器学习的发展,而逐渐趋向成熟的后者又反过来对渲染产生了积极影响。
在计算机视觉领域,考虑到精确标注的现实数据往往不易取得,近年来许多算法
利用渲染技术来生成人工数据,同样取得了良好的效果。
下面将对一些渲染中的机器学习算法和基于渲染的数据生成相关工作进行介绍,
限于篇幅,以下算法均只涉及基于路径追踪的渲染。

\subsection{用于渲染的机器学习算法}

正如2.1.1小节中所述,对于光线追踪的渲染优化往往从两个角度进行优化:
通过不断优化采样方式,或是对生成的图片进行后处理。
在与机器学习相关的算法中,与采样相关的工作较少,
其中Thomas Muller等人在2018年的工作中尝试利用神经网络生成路径追踪中的采样函数\cite{neuralImportanceSampling},在渲染结果上取得了显著的提升。
%https://cgl.ethz.ch/Downloads/Publications/Papers/2018/Mue18a/Mue18a.pdf

相比之下,更多的工作将目光集中在图片后处理上。
举例来说,Kalantari等人在2015年的工作中\cite{MachingLearning1},通过机器学习算法对一个非局部均值滤波器的参数进行训练,从而达到对生成图像过滤的作用;
Bako等人在2017年提出了一中基于机器学习的去噪器\cite{MachingLearningDenoiser},用于过滤采样数较高的结果图像。
同年,来自Nvidia公司的Alla Chaitanya等人利用CNN模型及Auto-Encoder模型
构建了用于采样数较少的图像的去噪器\cite{NvidiaDenoiser}。值得一提的是,
该模型后来也被整合进OpTiX库中,并在后来其发布的RTX实时光线追踪功能中被用于优化结果\cite{RTXDenoiser}。

\subsection{基于渲染的计算机视觉数据生成}

随着计算机视觉的发展,在一些领域如图像分割、图像语义以及三维重建中,获得真实图像中的精确(GroundTruth)数据往往需要通过人工标注的方式完成,
带来了巨大的工作量。随着渲染技术的日益壮大,许多研究着力于利用渲染的过程中附带的几何信息(如位置,法向量等)对生成的渲染图像进行标注,
从而批量生成可用的训练数据(也称合成数据)。在这些数据集中,有的通过光栅化渲染生成,有的也借助于光线追踪渲染(也被称作真实性生成数据);
有的专注于室内场景,而有的则主要以室外为主。

对于室内场景的数据,
主要有2014年Handa等人提出的ICL-NUIM数据集\cite{ICLNUIM}(2种房间布局,采用光栅化渲染),
John McCormac等人于2016年提出的SceneNet数据集\cite{SceneNet}(57种布局,采用光线追踪渲染),
Yinda Zhang等人于2017年提出的SUN CG数据集\cite{SUNCG}(45000余种布局,同时采用两种渲染),
以及2018年由Wenbin Li等人提出的InteriorNet数据集\cite{InteriorNet}(2200000余种布局,同时采用两种渲染)等等;
而对于室外的数据,则主要有
2016年German Ros等人提出的SYNTHIA数据集%\cite{The SYNTHIA Dataset: A Large Collection of Synthetic Images for Semantic Segmentation of Urban Scenes},
Adrien Gaidon等人于2016年提出的Virtual KITTI数据集%\cite{Virtual Worlds as Proxy for Multi-Object Tracking Analysis},
等人提出的UnrealCV数据集%\cite{UnrealCV: Connecting computer vision to unreal engine},
Magnus Wrenninge等人于2018年提出的SynScape数据集%\cite{Synscapes: A Photorealistic Synthetic Dataset for Street Scene Parsing}
等等。

总体而言,随着时间向后推移,新的数据集往往有着更大的数据量,更高的图像清晰度以及渲染质量。对于数据生成者来说,
这就对渲染器的要求也就越加苛刻。事实上,有不少的工作(如SceneNet,InteriorNet等)同样采用了GPU路径追踪渲染,
为本文的研究也提供了一定帮助。