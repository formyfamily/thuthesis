\chapter{架构设计}
\label{cha:pipeline}

本章将主要对本人设计的GPU路径渲染平台进行架构方面的介绍。
首先,由于该平台建立在OpTiX的基础之上,因此会对该库进行简要的介绍。
接着,会从总体的角度阐述平台的框架设计思路以及类结构,
然后再逐一对框架的各个组成部分进行分析。

\section{OpTiX库介绍}

由于OpTiX是由Nvidia开发,因此其本质是一个基于CUDA通用并行语言上的API库。
下面的这张流程图,直观地展示了其内部的工作原理:

TODO: optix framework figure

在上图中,黄色的方框代表需要用户自己编写的CUDA程序,蓝色和灰色的部分则由OpTiX实现。
在用户通过rtContextLaunch开始渲染后,GPU会对其各个处理单元分别调用入口函数Ray Generation Program,并分配各自不同的索引。
该函数需要按照索引(通常和对应像素的坐标值一致)向场景中发射光线,接着调用rtTrace函数接口开始光线追踪流程。

对于每一根光线,OpTiX会在描述场景的结构图(Node Graph)中进行遍历,从而找到与之相交的物体。
在这其中,Node Graph Traversal负责递归查找,Acceleration Traversal负责利用包围盒等算法为求交过程加速,
而Selecor Visit Program则是由用户添加的一些遍历规则(本文中没有使用)。
如果某条光线和场景中一个物体的包围盒相交,系统便会进一步进入Intersection Program阶段,判断光线与物体是否相交。
如果是,系统会将该相交事件进行记录,同时调用一次与该物体相关的Any Hit Program。

在所有物体都完成遍历之后,如果没有任何相交出现,系统会调用一次Miss Program,否则针对最近的相交再调用一次Closest Hit Program。
在以上的三者中(对应图中的Shade部分),都可以再次调用rtTrace函数,从而进入光线追踪的下一层递归。
最后,如果在以上的任何步骤中出现了诸如栈溢出、违法防存的情况,则会立即终止所有程序,并直接调用Exception Program。
关于函数调用的情况大致便是这样。

除了运行流程外,还需要对OpTiX的几个重要数据结构(以OpTiX Prime++为准,在文档中这些结构又被统称为Handle)进行介绍。

首先要介绍的是Context(上下文)类。顾名思义,Context类是Host程序(即通过CPU运行的C++程序)和内核程序(运行在GPU上的CUDA程序)之间的桥梁。
Host程序需要将运行所需的所有数据、函数信息一并输入进该类的实际对象中,然后再通过调用这一对象的launch方法开始渲染。
在上述的几个CUDA程序中,Ray Generation Program和Miss Program是与Context直接进行绑定的。

其次是GeometryInstance(几何物体)类和Material(材质)类,前者表示场景中的各个物体,后者则对应着这些物体的材质。
每个GeometryInstance对象在创建时都需要绑定一个Intersection函数和Bound函数,分别用于该物体的求交计算和包围盒计算。
GeometryInstance对象还需要绑定一个或多个Material对象,并通过Intersection函数确定具体使用哪种材质进行计算。
另一方面,对于Material类而言,则需要绑定一个Closest Hit Program函数和一个Any Hit Program函数。
当判定相交的Intersection函数确定采用某一材质时,便会在之后调用它所绑定的这两个程序。

本文还使用到了GeometryGroup,Transform,Buffer,Program等许多其他数据结构,但在这里不再详细进行介绍。
最后,还需要说明一下OpTiX中的数据组织方式。

对于上述介绍的三种类型(Context,GeometryInstance,Material),除去由OpTiX规定的输入参数外,用户还可以添加其它所需的自定义数据。
这类数据的组织方式与Python语言中的Dict类似,采用字符串下标的哈希表维护。
用户可以输入的自定义数据格式包括但不限于:基本类型(int, float等)、向量类型(float3, matrix等)、
缓存数据(Buffer类)、以及自定义的函数指针(Program类)。不得不提的是,这其中的函数指针部分使得我们得以在最终的架构内实现多态。
这对本人实现渲染器的可拓展性起着决定性的作用。

\section{总体架构}
 
从需求上来说,本人期望实现的渲染器能够支持以下几个主要功能:

\begin{itemize}
    \item{能读取不同种类的场景文件}
    \item{能够支持多种不同的几何体、材质类型、摄像机以及渲染算法}
    \item{拥有图形界面,实现交互式渲染}
    \item{支持动态调整场景功能,并能够实现动画}
    \item{可以进行分布式渲染}
\end{itemize}

根据这些要求,本文在先后参考了PBRT\cite{PBRT},及由Nvidia发布的OpTiX样例程序后,
最终设计了如下图所示的框架结构(包括控制流、数据流):

TODO: framework control / data flow figure

总体而言,该框架一共由四个模块组成:\textbf{Scene},\textbf{Renderer},\textbf{Post Processor}以及\textbf{GUI}。其中,
\textbf{Scene}的功能是将不同种类的输入场景文件进行读取,以及将数据转化成可被OpTiX使用的中间格式;
\textbf{Renderer}的功能是建立并配置OpTix中的Context类,将Scene中的所有数据输入到Context中,以及调用launch方法完成渲染工作;
\textbf{Post Processor}的功能是对Renderer的输出图像进行后处理,以及不断向Renderer提供反馈信息;
最后,\textbf{GUI}负责图形界面部分,将处理出来的图像展示到屏幕中,并处理由用户传来的交互数据。

除了模块结构外,同样还需要设计一套完整的工作流程。本框架制定的流程共分为6步:

\begin{enumerate}
    \item{用户启动渲染器,制定输入场景文件以及全局参数;}
    \item{根据输入的场景文件格式,建立Scene模块对文件进行解析;}
    \item{根据输入的全局参数,建立Renderer,并初始化Context类;}
    \item{将Scene中的数据,以及全局参数提供给Context;}
    \item{建立GUI,初始化渲染结果,然后开始以下循环:}
    \begin{enumerate}[\arabic{enumi}.1]
        \item{启动Renderer完成一轮渲染,并将结果进行累加;}
        \item{将累计结果输入Post Processor进行后处理;}
        \item{显示Post Processor的输出图像,同时向Renderer提交反馈;}
        \item{监控用户操作,如果场景或参数出现改动,则跳至第4步;}
    \end{enumerate}
    \item{保存结果,退出程序。}
\end{enumerate}

以上便是关于该架构的总体情况。下面将按照四个模块在流程中出现的顺序,依次对它们的设计细节进行阐述。
需要注意的是,后面的内容为了可能会为了功能扩展而对前面的结构进行修正,
因此最终的实现以本章结束时所得的版本为准。

\section{场景读取与载入}

TODO:Scene内所有结构的综合UML图

载入场景往往是一般渲染流程中工作量最为庞大的步骤之一,而对于GPU渲染器来说,这一步则显得更加困难。
这是因为,在从输入文件中读取数据的同时,我们还需要考虑如何将这些数据有效地组织起来,
从而使得GPU中的程序能够满足多态性,对各种各样的场景都能提供支持。
为了达到这点要求,我们首先便需要分析支持的场景应当由哪些部分组成:

\begin{itemize}
\item{各个物体的几何模型}
\item{物体上不同材质的属性(包括贴图)} 
\item{场景中不同介质的属性} 
\item{光源属性}
\item{摄像机的属性以及位置}
\item{以上所有属性随时间的变化情况(动画属性)}  
\end{itemize}

在OpTiX内的渲染流程中,使用这些属性的程序各有不同。
物体的几何模型在物体的两个求交函数中被用到;
材质属性、介质属性以及光源属性会在Closest Hit、Any Hit以及Ray Generation三类程序中得到使用;
摄像机的属性只会被用于Ray Generation程序;
动画属性仅仅被Host(CPU)所使用,因此不涉及任何GPU程序。
根据这些属性的应用范围,本架构提出了三种不同的抽象类\textbf{Geometry、Property、Animator}对它们进行封装。
其中,\textbf{Geometry}类用于几何模型;
\textbf{Property}类用于几何外的所有静态属性——材质、介质、光源、摄像机;
\textbf{Animator}类则用于动画属性。

\subsection{几何模型——Geometry}
TODO:Geometry的UML类图

由于我们的最终目标是将数据输入OpTiX中,所以可以反过来,从OpTiX的角度讨论如何设计Geometry。
对于OpTiX而言,建立一个GeometryInstance对象需要提供三种数据:
两个用于求交的函数(Program)、描述几何模型的参数(自定义参数)以及物体的材质(Material)。
可以发现,用于求交的函数往往只与Geometry的种类相关,因此这里看作是Geometry子类的一个静态变量。
几何模型的参数直接保存在各个实例中。
至于物体的材质,我们并不在Geometry中进行维护,而从调用它的地方直接取得。

综合上述,Geometry向外提供了一个直接生成GeometryInstance的方法buildGeometryInstance。
另外,在一些场合中可能需要对Geometry进行变换操作,因此还需要提供一个applyTransform方法。

\lstset{language=C++}
\begin{lstlisting}
virtual GeometryInstance buildGeometryInstance(Context&, vector<Material>&);
virtual void applyTransform(Transform&);
\end{lstlisting}

\subsection{场景属性——Property}
TODO:Property的UML类图

Property类是在场景输入当中最为重要的一环,它所解决的问题,是如何将不同种类的场景属性按照尽可能一致地输入进OpTiX中,从而使得所需的代码量达到最少。
这些属性往往由两部分组成,第一部分是数据,另一部分则是不同属性种类对应的计算函数。
举例来说,对于物体的材质而言,GPU中的内核程序既需要知道材质的各种参数(如颜色、反射率、光泽度等),
还得知道如何利用这些参数计算BRDF值,如何进行采样。
然而在GPU程序中,既不能通过指针参数的方式做到数据多态性,也不能通过虚函数的方式实现函数重载。
所以,在这里只能采用更加初级的方法来解决问题。

关于数据部分,本人最终采用的是通过OpTiX的Buffer输入给GPU的方法。
OpTiX的Buffer能够以数组的形式向内核程序传递任意类型的数据,但要求数组中所有元素的格式需要保持一致。
因此,这里采用了填充(Padding)的方式来对齐数据。定义Property的子类PaddingProperty,
其大小必须超过所有Property中的最大长度,来作为Buffer的数据类型。
对于内核程序而言,在拿到一个PaddingProperty类时,只需要知道它的实际类型(包含在Property类中),然后进行一次格式强制转换,便可以获得想要的数据了。

事实上,相较于上述的方法,还有一种看起来似乎更加直接的思路:和Geometry同样,利用OpTiX中添加自定义数据的方式来完成输入。
但是,本人最终没有采用这种方式,理由主要有两点(以材质属性为例):第一,如果采用这种方案,材质属性应该被绑定在和它对应的Material中。
然而在一些渲染算法如SPPM中,Ray Generation程序也需要使用到这些属性,但由于该程序无法直接访问Material,因此这种方案并不能提供相应的支持;
第二,如果采用了这种方案,则不同种类的Material必须配置各自的Closest Hit Program,来使用该类型的属性,
这便使得代码量变得冗长许多。而相比之下,上面的方法只需要用到一个程序即能达成目标。

这就涉及到关于Property设计的第二部分——如何重载属性的计算函数。这里本框架采用的是函数指针的方式。
OpTiX中的函数指针实际是通过函数ID(rtCallableProgramId)来表示的,这个ID同样是Buffer支持的数据类型之一。
在渲染开始之前,Host程序会对所有可能出现的计算函数都计算函数ID,然后保存在对应的Buffer之中;
到了渲染时,内核程序只需要知道属性的类型,便可以在Buffer里面查找对应的函数,然后直接进行调用。

根据四种不同的属性,Property一共拥有四个子类:\textbf{MaterialProperty},\textbf{LightProerty},\textbf{CameraProperty}以及\textbf{MediumProperty}。
这四者在结构上基本保持一致,它们又拥有着属于各自的子类。限于篇幅,有关这些不同Property的内部实现,这里便不再进行逐一介绍。

最后需要提到的一点是,和Geometry不同,Property作为数据结构,需要直接出现在内核程序之中。
然而,将这些Property输入OpTiX的方法(比如说生成Material)却涉及到OpTiX的Host代码,因此不能被引入内核程序,这就引发了矛盾。
为了解决这个矛盾,本人又加入了一个新的类PropertyManager,来封装所有需要Host程序完成的工作。PropertyManager一共提供了三个函数接口:

\lstset{language=C++}
\begin{lstlisting}
void importPropertyBuffer(Context);
void changePropertyBuffer(int);
virtual void importPropertyProgram(Context);
\end{lstlisting}

其中,importPropertyBuffer函数将已有的Property(保存在PropertyManager中)通过填充方法放入Buffer之中;
changePropertyBuffer函数将Buffer中指定下标的Property的数据进行更新;
而importPropertyProgram函数则负责把与Property相关的所有计算函数输入OpTiX中。
和Property一样,PropertyManager也有着相应的四个子类,但这些子类之间有着更大的差异(如LightPropertyManager还会维护光源的Geometry)
且都没有被进一步继承下去。

\subsection{动画——Animator(未实现)}
TODO:动画

\subsection{场景——Scene}
Scene类表示的是整个场景中的所有信息。从之前列举的场景内容来看,
它需要包含一个Geometry组成的数组、四种不同的PropertyManager,和一个Animator。
在功能上,它应当支持两个操作:从文件中读取数据,以及将数据放入到OpTiX的Context中。这两个操作对应的接口函数为:
\lstset{language=C++}
\begin{lstlisting}
virtual void loadScene(Context);
virtual void importSceneToContext(Context, vector<Program>&);
\end{lstlisting}

这里有两点需要做出特殊说明。首先,从设计上来说,在读取文件时不应当需要使用Context作为参数,
但是这里加入了Context,主要是为了在一些地方(如Mesh读取出)提前获取Buffer的地址,从而避免在读取完后再进行数据拷贝;
其次第二个函数的参数中加入了一个Program数组,这些Program实际是由之后介绍的Renderer所确定的Closest Hit Program和Any Hit Program,
由于这两者必须被绑定到由Scene生成的Material上,所以这里它们也需要作为参数传进来。

\section{渲染器——Renderer}
TODO:Renderer的UML类图

Renderer类的主要任务是生成并初始化Context,及完成渲染工作。事实上,有了之前Scene的基础,它在实现上相对简单许多。
Renderer需要维护的数据包括但不限于由它生成的Context,与渲染算法相关的Hit Program,以及所有用于渲染的输入、输出Buffer。
此外,它还需要支持以下的四个接口函数:

\lstset{language=C++}
\begin{lstlisting}
virtual void generateContext();
virtual void uploadScene(Scene&);hou
virtual void resize(uint, uint);
virtual void launch();
\end{lstlisting}

generateContext函数的功能便是是建立Context。为了让Context能够运转,在该函数中还要根据不同的渲染算法,
设定Context的全局参数、创建输入输出的Buffer、以及为Context指定Ray Generation、Exception和Miss三种程序。
另外,该函数还会生成用于后方的两种Hit Program。

uploadScene函数的主要作用的是调用上述Scene中的importSceneToContext函数来完成场景读入,但除此之外,
一些需要根据Scene的hui you(如包围盒大小)来决定的Renderer参数也会在这里进行设置。

resize函数用来修改渲染器的尺寸,在图形界面中支持用户对窗口进行放缩的操作。
具体内容主要包括修改相机参数、各种图像Buffer的尺寸大小等等。

最后也是最为重要的是函数launch,其功能便是用OpTiX完成一轮渲染。
在这个函数中,将会一次或多次调用Context的launch接口,启动OpTiX的GPU渲染程序。
在下一章GPU光线追踪渲染算法的介绍中,对这个函数内部的实现会展开更细致的说明。

\section{图像后处理——PostProcessor}

TODO:PostProcessor

PostProcessor负责将Renderer的输出图形进行后处理,然后再将一些信息反馈回Renderer。
他的构造非常简单,由一个输入Buffer,一个输出Buffer,以及所有的辅助数据(包括反馈、由渲染器提供的其他信息)组成。
它需要实现三个接口函数:

\lstset{language=C++}
\begin{lstlisting}
virtual void setupWithContext(Context&);
virtual void resize(uint, uint);
virtual void process(Context&);
\end{lstlisting}

简单来说,setupWithContext函数负责初始化处理器,resize函数和之前一样用于修改尺寸,
而process函数便是完成一次后处理操作。用户创建一个后处理器后,首先要手动制定其输入Buffer,然后通过初始化
便会自动生成输出Buffer和其它的辅助数据。这些数据可以作为最终输出,可以接回Renderer,也可以再接到后续的后处理器上。

另外需要注意的是,对于有反馈的后处理算法而言,Renderer也需要做出相应的修改以提供支持。
一种比较合理的解决方法是采用Decorator设计模式,通过对Renderer提供修饰来达到这一目标。
但由于渲染算法的修改通常都需要.深入到OpTiX的内核程序中去,因此在这个过程中有许多代码是不可避免地需要重写的。


\section{图形界面——GUI}

TODO:GUI UML图

GUI类是整个流程当中的最后一个被创建的模块,负责与用户之间的一切交互。
它主要需要完成三个功能:运行完整的渲染流程、现实结果、以及允许用户进行修改。

GUI类拥有一个Scene的指针,一个Renderer的指针
以及一个由PostProcess的指针组成的数组。这三者包含了有关渲染的全部信息。
它的主要接口函数有两个:
\lstset{language=C++}
\begin{lstlisting}
void initialize();
void run();
\end{lstlisting}

用户首先需要调用initialization函数初始化GUI(主要是初始化其底层支持的库如glfw等),
然后将生成的Scene、Renderer等其他模块逐个输入进GUI中,
最后调用run函数开始循环。

GUI类的具体实现是基于Dear ImGui图形界面库\cite{Imgui}(以下称为ImGui)完成的。该界面库拥有扩展性强,简单轻便的特点,
因此很方便用来实现用户的交互操作。在该框架中,用户一共可以进行四种类型的操作:修改窗口的大小;修改当前时间(用于动画);
通过鼠标拖拽修改相机位置;通过表项修改场景属性(或渲染属性)。
关于修改窗口大小、修改时间的操作,只需要直接调用之前介绍的方法即可完成。
其余两种操作则在实现上更为复杂。

首先说如何修改相机位置。当ImGui检测到鼠标事件后,后台需要根据事件的具体情况来计算出新的相机位置,然后将其输入到OpTiX中。
可以发现,CameraPropertyManager类正好适合于完成这样的工作。因此,在该类中又增加了一个新的接口函数:

TODO:看看这里接口对不对
\lstset{language=C++}
\begin{lstlisting}
void processMouse(float, float, bool, bool, bool, float);
\end{lstlisting}

该函数的参数均为描述鼠标事件的变量。在计算完新的相机信息后,它只需直接调用changePropertyBuffer函数便完成了对OpTiX的输入。

其次来说场景属性的修改。对于这一部分来说,主要的问题是如何为不同的属性设置表项。
这里采用的方法是:通过递归算法遍历场景中每个属性对应的实体对象,然后由这些对象自行向ImGui提出所需的表项格式。
为此,之前提到的所有类(Geometry、Property、PropertyManager、Scene、Renderer、PostProcessor)都需要再添加一个接口函数:
\lstset{language=C++}
\begin{lstlisting}
virtual bool displayGuiTab() 
\end{lstlisting}

该函数会在其内部调用ImGui的方法,而后者则会对这些调用统一进行记录。最终,ImGui会将所有被记录的表项都显示在屏幕上。
函数需要返回一个Bool值,表示该表项是否被修改过,以决定是否要对OpTiX上的数据进行更新。

最后,下图展示了该框架下所有类型之间的关系:
TODO:综合UML图